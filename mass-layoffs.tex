\documentclass[12pt,letterpaper]{article}
\usepackage[margin=1in]{geometry} % Customize paper layout.
\usepackage{lmodern}              % Load font.
\usepackage{titling}              % Customize title.
\usepackage{parskip}              % Add space between paragraphs.
\usepackage{enumitem}             % Customize list enumeration.
\usepackage{amsmath, amssymb}     % Extend equation environments.
\usepackage{threeparttable}       % Extend table environment.
\usepackage{siunitx}              % Typeset values with units.
\usepackage{multirow}             % Merge table rows.
\usepackage{graphicx}             % Support figures.
\usepackage{float}                % Place figures in specific location.
\usepackage{adjustbox}            % Scale and rotate tables and figures.
\usepackage{rotating}             % Rotate tables.
\usepackage{caption}              % Customize table and figure captions.
\usepackage{color, soul}          % Support highlighting.
\usepackage{apacite}              % Support APA style citations.

% Write the title block.
\title{Mass layoffs and college enrollment in Maine}
\author{Patrick Lavallee Delgado}
\date{21 December 2023}

\begin{document}

% Format regression table.
\sisetup{
  detect-mode,
  group-digits            = integer,
  group-minimum-digits    = 4,
  group-separator         = {,},
  round-mode              = places,
  round-precision         = 3,
  round-pad               = false,
  input-signs             = ,
  input-symbols           = ,
  input-open-uncertainty  = ,
  input-close-uncertainty = ,
  table-align-text-before = false,
  table-align-text-after  = false,
}

% Remove space above equations.
\setlength{\abovedisplayskip}{0em}

\maketitle

\section{Introduction}

Mill towns were once centers of productivity throughout northern New England, but have long been in decline as the mills they were built around have closed. A job at the mill did not require education beyond high school, offered long and stable employment, and created a tradition and expectation around a way of life. Similar jobs have not replaced those lost at the mills. Changes in regional labor markets, like those seen in northern New England mill towns, can change requirements for educational attainment. But any number of barriers may prevent individuals, particularly new entrants to the regional labor market, from pursuing additional education beyond high school. Perhaps a massive shock can induce change, otherwise a policy intervention may be required to promote economic development and college access for those who need it most.

This study examines the relationship between mass layoffs and college enrollment among high school students in Maine, a northern New England state where mill towns were once the driving economic force throughout its rural interior. I propose a flexible framework to model the accumulation of mass layoff signals that students perceive at their respective schools, proportional to the relative severity and proximity of the event. This approach accounts for how a relatively rare but significant event can have variable but non-negligible impacts on students' decision-making across the state. The findings can identify an opportunity for public policy to address the educational attainment needs of students in mill towns who continue to struggle amidst long economic decline and depreciating human capital.

\section{Conceptual framework}

Students about to graduate high school face a choice for what they do next. This is the postsecondary transition, at which a student chooses either to enter the workforce or enroll in college. Classical human capital theory provides that a student would choose whichever maximizes their expected earnings and well-being. So, a student would weigh the benefits and costs for their particular situation and idiosyncratic preferences, choosing to enroll in college if the expected benefits exceed the upfront costs. Extensive research confirms that additional investments in education beyond high school are associated with better jobs, higher lifetime earnings, better health, and higher quality of life \cite{eide-showalter:2010}. While a college degree produces the largest benefits, any number of college credits offers a significant advantage relative to a high school education alone \cite{kane-rouse:1995}.

At the same time, postsecondary education imposes significant costs, including tuition for college and foregone earnings from time spent in school instead of the workforce. Some evidence suggests that college graduates can make up the difference in the long run, but the ballooning cost of college and the current student debt crises undermine the generalizability of that finding to future college students. Existing financial aid programs can help students pay for college, but generally not the cost of living or transportation without additional borrowing. Recent research demonstrates that students are not clear on which academic programs lead to better jobs and overestimate the value of chosen field. Students may face immediate family obligations to start providing or otherwise have stronger present biases for earning and income upon graduation. Taken together, incomplete information about college and uncertainty of its returns can make the investment appear too risky to pursue \cite{plank-davis:2010}.

Society also has an interest in the choice that student makes. Research shows that education, especially quality education, develops cognition and knowledge in a workforce to become more productive, specialized, and valuable \cite{hanushek-wößmann:2010}. Higher average educational attainment is associated with stronger economic growth, lower poverty rates, lower public welfare costs, and higher tax receipts \cite{mcmahon:2010}. Public policy attempts to capture these external benefits to education by incentivizing individuals to pursue its private benefits. On the other hand, no new investment in education lets skills depreciate over time with consequences for wages and growth. It follows that both individuals and societies must reach for higher levels of educational attainment to remain competitive and to conitnue realizing the private and societal returns to education.

However, students in mill towns perceive higher costs and lower benefits with respect to college. College may be less accessible because of distance and lack of financing, especially for the cost of living while in school. Even if these barriers were surmountable, jobs that require a college degree may not exist in their communities. Longstanding tradition and expectation almost surely weigh heavily in this choice, too. Work in mill towns had historically been dominated by low-skill manufacturing jobs that provided stable employment, a middle class standard of living, and an identity around that way of life. Despite the long and steady decline of the manufacturing sector, and mill towns with it, enrolling in college may not be understood to maximize expected earnings and well-being. Indeed, it remains uncommon for residents of these areas to pursue postsecondary education, let alone attain a college degree.

Consider the postsecondary transition from the perspective of students in a northern New England mill town. They likely have family and friends who lost their jobs at the mill. If the mill is even still operating, it probably does not feel like it will be for much longer. There are not many other jobs around town that would lead to the modest standard of living that their parents were able to attain, as these, too, only existed as a result of the economy that the mill sustained. Their parents likely did not go to college, and even if they are encouraged to go, it is not clear what it would lead to or how it would get paid for. They may know a couple people who went away to college, but one dropped out because it was too expensive, too difficult, or too far away; whereas another never came back. Maybe things will pick up again at the mill, after all, that is how everyone else got by.

From the point of view of an economist or policymaker, these students are likely to underinvest in their education because the benefits are too uncertain and the costs are too high. As a result, they produce none of the private or social returns to postsecondary education. And interestingly, the long decline of mill towns appears to be a sort of steady state in which individuals respond to change in the regional labor market with apparent inertia. Meanwhile, poverty rates increase, public welfare costs increase, and tax receipts decrease. Evidence that characterizes whether and how students weigh information about the regional labor market at the postsecondary transition can inform public policy to produce returns to education that private action alone does not.

Mass layoffs are a strong signal against pursuing a career in the affected industry. Students at the postsecondary transition may make a different choice than they otherwise would with additional information about the regional labor market. Since the existing economic decline has not induced a noticeable change in the average educational attainment in mill towns, perhaps a particularly destabilizing event like a mass layoff, one that students may even experience through family and friends, would change this trend. Specifically, a mass layoff that affects manufacturing may challenge the expectation for a job at the mill or one otherwise attached to the economy build around it, such that entering the workforce upon graduation suddenly appears riskier relative to enrolling in college. Whether a mass layoff is a strong enough nudge to induce a shift towards postsecondary education has equity implications for regional economic development and college access.

This study examines the relationship between mass layoffs and college enrollment among high school students in Maine, a northern New England state where mill towns were once the driving economic force throughout its rural interior. Previous research has considered the effect of mass layoffs on the regional labor market \cite{foote:2019} and as it relates to college enrollment \cite{acton:2021,foote-grosz:2020} at the county level. However, these analyses assume the flow of information about mass layoffs and the related macroeconomic conditions do not spill over arbitrary political boundaries. Instead, the present study attempts to model the accumulation of mass layoff signals that students perceive at their respective schools, proportional to the relative severity and proximity of the event. For example, a school would be equally receptive to the signal from a small mass layoff nearby as to that from a large mass layoff far away. Of course, this assumes that students from across the catchment area of the school are similarly aware of disruptions to the regional labor market. But to the extent that they are, either by word of mouth or through news media, this information get disseminated through social networks developed at the same school. This approach accounts for how a relatively rare but significant event can have variable but non-negligible impacts on students' decision-making across the state.

\section{Data}

I measure students' exposure to major shocks to the regional labor market using data on mass layoffs reported under the federal Worker Adjustment and Retraining Notification (WARN) Act. The WARN Act, as amended by Maine law, requires employers with at least 100 employees to give at least 90 days' notice to employees ahead of either (1) a plant or operating unit closure that affects 50 or more employees at a single site or (2) a mass layoff that affects 500 or more employees at a single site or 50 to 499 employees who together represent a third of employees at that site \cite{maine:warn}. These data describe the company, street address of the site, number of employees affected, and the date of notice. Where location information is missing, I find news articles that corroborate the event and identify the site. I also supplement these data the North American Industry Classification System (NAICS) code that best describes the business at the affected site. I then use a geolocating service to map the street address of each site onto the geographic coordinate system by latitude and longitude. The data describe 48 mass layoff events, 65\% of which affect manufacturing.

I observe students' college enrollment at the graduating cohort level using data from the National Student Clearinghouse (NSC).\footnote{These data are the NSC StudentTracker for High School reports, which the Maine Department of Education publishes online.} These data count the number of high school students in each graduating cohort who enroll in college within six, 12, and 24 months of graduation. I join these data onto school and graduating cohort characteristics in the Common Core of Data (CCD) and EDFacts, both data products of the US Department of Education that describe the universe of public schools in the country. These characteristics vary by school year and include the latitude and longitude of the school, enrollment and demographic composition of 12th graders, and math and reading proficiency rates of the high schools.\footnote{EDFacts does not disaggregate math and reading proficiency rates for high school grades.} I also join onto census tract characteristics from the American Community Survey (ACS), five-year estimates, from the US Census Bureau. These characteristics describe the educational attainment, unemployment rate, median household income, and average commuting time for residents in the census tract of the school. I use average commuting time as a proxy for average commuting distance.\footnote{The intuition here is that commuters' average driving speed over streets and highways is 60 miles per hour. This likely overstates the commuting distance is urban areas and understates it for rural areas.}

I successfully match all public high schools and graduating cohorts in the NSC data to their corresponding schools and years in the CCD and EDFacts data. Importantly, the NSC data describe graduating cohorts at each high school as far back as the 2010-11 school year, but only for schools extant as of the 2018-19 school year. In other words, the data systematically miss college enrollment for graduating cohorts from schools that had since merged or closed. I discard 23 private high schools in the NSC data that are not represented in the CCD census of public schools. I also discard two alternative schools and 24 career and technical education schools and that are not represented in the NSC data, a potentially problematic limitation if students in these schools are more likely to enter the workforce in pursuit of jobs affected by mass layoffs. The schools that remain in the sample are all regular public schools, including charter and magnet schools.

The primary innovation of this study is that the intensity of the signal from a mass layoff is proportional to the relative size and proximity of the event. First, I weight the event by its relative intensity in terms of the size of the civilian workforce proximal to the school. For example, a mass layoff even of arbitrary size has a larger impact on schools where the size of the regional workforce is small and a smaller impact on schools where the size of the regional workforce is large. Next, I weight the event by its relative distance in terms of the approximate average commuting distance proximal to the school. For example, a mass layoff equally distant from two schools has a larger impact where the average commuting distance is large and a smaller impact where the average commuting distance is small. The size of the signal is the sum of events weighted for each school in the year leading up to observations of college enrollment at six, 12, and 24 months after graduation. Formally, this is defined as

\begin{align}
  \text{shock}_{ct} = \sum_{i=1}^{N_{ct}}\left(\frac{\text{layoffs}_{i}}{\text{workforce}_{st}}\right)\left\lceil\frac{\text{distance}_{is}}{\text{commute}_{st}}\right\rceil^{-1} \quad t \in \{6, 12, 24\}
\end{align}

where (1) $\text{shock}_{ct}$ is the signal for cohort $c$ at follow up $t$, (2) $N_{ct}$ is the total number of events in the year before follow up, (3) $\text{layoffs}_{i}$ is the number of workers dislocated by event $i$, (4) $\text{workforce}_{st}$ is the size of the civilian workforce in the census tract of school $s$ at follow up, (5) $\text{distance}_{is}$ is straight line distance between the school and the event, and (6) $\text{commute}_{st}$ is the approximation of the average commuting distance in the census tract of the school at follow up. To calculate this measure, I join all graduating cohorts onto all mass layoff events, calculate the weights for each pairwise combination, then aggregate back to the cohort and follow up level. The final dataset contains 865 graduating cohorts from 127 public high schools observed at six, 12, and 24 months after graduation for a total of 2,595 observations. These cohorts span the 2013-14 and 2019-20 school years, the period over which both WARN and NSC data are available.

\section{Empirical strategy}

I estimate the effect of mass layoffs on college enrollment, separately for follow ups at six, 12, and 24 months after graduation. The primary specification takes the form

\begin{align}
  y_{c|T=t} = \delta\text{shock}_{c|T=t} + \boldsymbol{X}_{c|T=t} + \boldsymbol{\gamma}_{s} + \boldsymbol{\tau}_{c} + \varepsilon_{c|T=t} \quad t \in \{6, 12, 24\}
\end{align}

where (1) $y_{c|T=t}$ is the enrollment rate in any college, two-year college, or four-year college for cohort $c$ at follow up $t$, (2) $\boldsymbol{\delta}$ is the partial effect of mass layoffs in the year leading up to follow up $t$ and is the coefficient of interest, (3) $\boldsymbol{X}_{c|T=t}$ is a vector of graduating cohort and school census tract demographic characteristics at follow up $t$, (4) $\boldsymbol{\gamma}_{s}$ is a high school fixed effect, (5) $\boldsymbol{\tau}_{c}$ is a graduation year fixed effect, and (6) $\varepsilon_{c|T=t}$ is the idiosyncratic cohort error term at follow up $t$. I calculate robust standard errors clustered at the high school level.

The fixed effects attempt to control for unobservable differences in enrollment across high schools and across graduation years. This strategy assumes that there are no changes in unobserved determinants of college enrollment within high school or within graduation year that are correlated with mass layoffs. This assumption might not hold if preferences for college enrollment change over time within a high school, or if contemporaneous shocks induce differential college enrollment across high schools within a graduation year.

\section{Results}

Table \ref{table:primary} reports the results of the model applied to each follow up (columns) and college enrollment outcome (panels). Panel A suggests that mass layoffs had no impact on enrollment in college at six, 12, or 24 months after graduation. The breakdown by college type offers more nuance. Panel B suggest the mass layoffs signal is associated with a 13 percentage point increase in enrollment in two-year college at six months after graduation. Panel C shows a 27 percentage point decrease in enrollment in four-year college at six months after graduation and a 13 percentage point decrease at 24 months after graduation. The measure of the signal allows a flexible interpretation of this finding in terms of the relative size and proximity of the event. So more precisely, the sum total of mass layoffs equivalent to one event the size of the civilian workforce and within the commuting distance of the census tract of the school is associated with effects of these sizes.

\begin{table}[!htbp]
  \centering
  \begin{threeparttable}
    \caption{Effect of mass layoffs on college enrollment}
    \label{table:primary}
    \begin{tabular}{lSSS}
      \hline\hline
      & \multicolumn{1}{c}{6 mos.} & \multicolumn{1}{c}{12 mos.} & \multicolumn{1}{c}{24 mos.} \\
      \hline \\
      & \multicolumn{3}{c}{A: Enrolled in any college} \\
      \cline{2-4} \\
      \input{out/primary-pct_coll.tex} \\
      \hline \\
      & \multicolumn{3}{c}{B: Enrolled in two-year college} \\
      \cline{2-4} \\
      \input{out/primary-pct_coll_2yr.tex} \\
      \hline \\
      & \multicolumn{3}{c}{C: Enrolled in four-year college} \\
      \cline{2-4} \\
      \input{out/primary-pct_coll_4yr.tex} \\
      \hline
    \end{tabular}
    \begin{tablenotes}
      \small
      \item \textit{Notes:} Robust standard errors clustered at the high school level in parentheses. Each cell is a separate regression of a college enrollment outcome on mass layoffs and covariate controls. Each observation is a graduating cohort.
      \item  $^{*}\text{p}<0.05$; $^{**}\text{p}<0.01$; $^{***}\text{p}<0.001$
    \end{tablenotes}
  \end{threeparttable}
\end{table}

These are very large shifts in college enrollment, but because there is no change in the overall share of students who choose to enroll in college upon graduation, these changes in behavior are likely among students who already had strong preferences to enroll in college. In other words, receiving a stronger accumulated mass layoff signal induces students to substitute four-year college for two-year college relative to those who receive a weaker one. Perhaps more concerning is that a stronger signal is also associated with lower persistence in four-year college. Of course, these cannot be investigated further without student-level data.

\section{Conclusion}

This study offers suggestive evidence of the relationship between mass layoffs and college enrollment among high school students in Maine. It finds that students do consider mass layoffs at the postsecondary transition, when they choose between entering the workforce and enrolling in college. Specifically, in the presence of a strong mass layoff signal, students substitute four-year college for two-year college or no college at all relative to those who receive a weaker mass layoff signal. This shift may indicate that the destabilizing shocks over an accumulation of mass layoff signals somehow changes their access to or risk assessment of four-year college. Interestingly, there is no change in the overall rate of students' investment in postsecondary education, but a decline in the quality of education that they demand. These findings identify an opportunity for public policy to address the educational attainment needs of students in mill towns who continue to struggle amidst long economic decline and depreciating human capital.

\bibliographystyle{apacann}
\bibliography{mass-layoffs}

\end{document}
